\subsection{Koordinatensysteme in der Astronomie}

In der Astronomie gibt es verschiedene Koordinatensysteme zur Beschreibung von Positionen am Himmel. Jedes hat seine Vor- und Nachteile, generell weit verbreitet sind das äquatoriale und das horizontale Koordinatensystem.
Das äquatoriale Koordinatensystem benutzt Rektaszension und Deklination um Positionen zu bestimmen und ist an der Erdachse orientiert und somit global gültig. Ein Vorteil ist, dass nur in der Achse der Rektaszension mitgeführt werden muss.
Im Gegensatz dazu ist das horizontale System standortabhängig und beschreibt Positionen relativ zum lokalen Horizont, was sehr intuitiv ist. Die Steuerung des NRTs erfolgt mechanisch über die beiden Achsen des horizontalen Systems - Azimut (AZ) und Elevation (El) - in der Software für den Client ist jedoch auch eine Umrechnung in das äquatoriale und auch das galaktische Koordinatensystem ersichtlich. 
Ein Nachteil ergibt sich durch die Verwendung des AZEL-Systems bei Beobachtungen nahe des Zenits, wo kleine Positionsänderungen zu langen Bewegungen in AZ führen.

\subsection{Mathematische Darstellung der Teleskopposition}

Um die Distanzen / Pfade zwischen Start- und Zielkoordinaten zu berechnen, werden die Punkte hierfür auf der Oberfläche der Einheitssphäre dargestellt. Dies dient zur späteren Verwendung für die Interpolation von Strecken auf der Kugeloberfläche.
Die Umrechnung von horizontalen zu kartesischen Koordinaten erfolgt durch die Verwendung trigonometrischer Funktionen (Kugelkoordinaten). Alle Umrechnungen erfolgen in Radianten / Bogenmaß, d.h. die AZ/EL-Koordinaten werden zunächst von Grad in Radiant konvertiert und dann für folgende Formeln verwendet.

\begin{equation}
    x = cos(EL) \cdot cos(AZ)
    \label{cartesian_x}
\end{equation}

\begin{equation}
    y = cos(EL) \cdot sin(AZ)
    \label{cartesian_y}
\end{equation}

\begin{equation}
    z = sin(EL)
    \label{cartesian_z}
\end{equation}
\\
Um die berechneten Zwischenpunkte für die Steuerung nutzbar zu machen, werden die kartesischen Koordinaten wieder in AZ/EL-Koordinaten mit folgenden Formeln rücktransformiert.

\begin{equation}
    hyp = \sqrt{x^2 + y^2}
    \label{hypotenuse}
\end{equation}

\begin{equation}
    AZ = arctan2(y,~x)~ \bmod ~360
    \label{arctan2_az}
\end{equation}

\begin{equation}
    EL = arctan2(z,~hyp)
    \label{arctan2_el}
\end{equation}
\\
Die Koordinaten sind grundsätzlich in Grad angegeben, die aufgezählten Formeln benötigen jedoch Radiant als Einheit der Winkel.
Für die Konvertierung von Grad (°) in Radiant (rad) wird folgende Umrechnung benutzt.

\begin{equation}
    radians = degrees ~\cdot~ \frac{\pi}{180}
    \label{degree to radians}
\end{equation}


\subsection{Sphärische Geometrie}

\subsubsection{Großkreis- vs. Azimutalpfad}

Zur Berechnung der kürzesten Strecke zwischen Start- und Zielpunkt wird zunächst die Verbindung entlang eines Großkreises betrachtet. Ein Großkreis ist ein größtmöglicher Kreis auf der Kugeloberfläche. Dadurch teilt er die Kugel in zwei gleich große Hälften und der Mittelpunkt stimmt mit dem Mittelpunkt der Kugel überein. Ein Beispiel dafür sind Längenkreise auf der Erde, währenddessen Breitenkreise (bis auf den Äquator) keine Großkreise sind. Großkreise sind in der sphärischen Geometrie das äquivalent von Geraden in der Ebene und damit die kürzeste Verbindung zwischen zwei Punkten.
Da die Fahrgeschwindigkeiten des NRTs unterschiedlich in Elevation und Azimut sind, ist der kürzeste Weg jedoch nicht immer der schnellste. Außerdem muss sich das Teleskop in der Elevation nur von 16° bis 90° (bzw. 164° mit Berücksichtigung des Flip-Overs) bewegen, da es sich nur in der oberen Kugelhälfte bewegen kann und durch den Boden limitiert ist.

Die Länge des Großkreisbogens lässt sich durch die zentrale Winkelentfernung delta_sigma berechnen:

\begin{equation}
    cos(\delta \sigma) = sin(EL_S) \cdot sin(EL_Z) + cos(EL_S) \cdot cos(EL_Z) \cdot cos(AZ_S - AZ_Z)
    \label{Winkelentfernung}
\end{equation}

\begin{equation}
    d = R \cdot \delta \sigma = arccos(cos(\delta \sigma))
    \label{Großkreisbogenlängenformel}
\end{equation}

Da jedoch die reine Länge des Kreisbogens für die Bewegung weniger relevant ist, dient diese nur als Schätzung für die erwartete Laufzeit. Für das Fahren des Teleskops werden Punkte entlang dieses Pfads interpoliert.

Während eine Möglichkeit die oben beschriebene Bewegung entlang eines Großkreises ist, ist die gleichzeitige Bewegung von Azimut- und Elevationsposition auch möglich. Ein Vorteil hierbei ist, dass weniger berechnet werden muss. Das Teleskop bewegt sich schlicht von Start- zu Zielazimut und selbiges gilt für die Elevationskoordinate. Diese Bewegung kann aufgrund der beiden separaten Motoren im Teleskop simultan stattfinden.

\subsubsection{SLERP-Interpolation \textendash~Vorteile und Grenzen}

optisch kürzester weg ? nicht immer der geometrisch kürzeste
slerp beschreibung
interpolation auf einheitskugel
warum sinnvoll (smooth + realistisch)






\subsection{Laufzeitberechnung}

\subsubsection{Einfluss von Azimut- und Elevationsgeschwindigkeit}
\subsubsection{Engstellen \textendash Limitierender Freiheitsgrad}

azel geschwindigkeiten
t max
langer achsenweg dominiert

\subsection{Mechanische Freiheitsgrade des NRT}

\subsubsection{Restriktionen durch Softwaregrenzen}
\subsubsection{Nutzung des vollen Bewegungsbereichs \textendash~Chancen \& Risiken}

auflistung + erklärung warum problematisch / vorteilhaft
darstellung

