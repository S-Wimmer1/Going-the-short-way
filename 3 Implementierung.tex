\subsection{Wegpunkte/Zielkoordinaten}

\subsubsection{Interne Darstellung und Genauigkeit}
\subsubsection{Umgang mit Singulärpunkten (z.B. Zenitnähe)}


\subsection{Koordinatentransformation}

\subsubsection{Live-Konvertierung im Steuerungsprozess}
\subsubsection{Optimierungsmöglichkeiten durch Vorberechnung}

\subsection{SLERP-Pfad}

\subsubsection{Mathematische Grundlage}
\subsubsection{Implementierungsdetails in Python/C++/C}

\subsection{axis aligned paths}

\subsubsection{Wann sinnvoll \textendash wann ineffizient?}
\subsubsection{Kollision mit Zenitregion}

\subsection{simultaneous az el path}

\subsubsection{Minimierung der Gesamtpfadlänge}

\subsection{Movement Time}

\subsubsection{Dynamische Geschwindigkeitsprofile}
\subsubsection{Worst-Case vs Best-Case vs. Average-Case}

\subsection{slerp path time}

\subsubsection{Vergleich mit linearer Interpolation}
\subsubsection{Vorteile bzgl. smoother Bewegung}

definitionen
umrechnungen
interpolation SLERP
alternative bewegungsstrategien
zeitberechnung